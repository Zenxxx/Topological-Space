\documentclass[dvipdfmx,a4paper, 12pt]{article}
\usepackage{wrapfig, blindtext}
\usepackage{amsmath,amssymb,graphicx} %load extra symbols and environments
\usepackage[margin=1in]{geometry} %set margins
\usepackage{enumerate}
\usepackage{mathtools}
\usepackage{autobreak}
\usepackage{graphicx}
\usepackage{color}
\usepackage{url}
\usepackage{tcolorbox}
\usepackage{bm}
\usepackage{breqn}
\usepackage{here}
\usepackage{ascmac}
\usepackage{empheq}
\usepackage{setspace}
\usepackage{here}
\usepackage{mathrsfs}
\usepackage{mathrsfs}
\usepackage{bbm}

\hyphenpenalty=10000\relax
\exhyphenpenalty=10000\relax
\sloppy

\newcommand{\R}{\mathbb{R}}
\newcommand{\Z}{\mathbb{Z}}
\newcommand{\N}{\mathbb{N}}
\newcommand{\Q}{\mathbb{Q}}
\newcommand{\C}{\mathbb{C}}
\newcommand{\T}{\mathbb{T}}

\allowdisplaybreaks[4]
\tcbuselibrary{breakable, skins, theorems}

\newtcbtheorem{definition1}{Definition}{enhanced,
attach boxed title to top left = {xshift=5mm,yshift=-3mm},
boxed title style = {colframe = blue!35!black, colback = white},
coltitle = black,
colback = white,
colframe = green!35!black,
fonttitle = \bfseries,
breakable = true,
top = 4mm
}{korehatheorem1}

\newtcbtheorem{theorem2}{Theorem}{enhanced,
attach boxed title to top left = {xshift=5mm,yshift=-3mm},
boxed title style = {colframe = green!35!black, colback = white},
coltitle = black,
colback = white,
colframe = orange!35!black,
fonttitle = \bfseries,
breakable = true,
top = 4mm
}{korehatheorem2}

\newtcbtheorem{proof3}{Proof}{enhanced,
attach boxed title to top left = {xshift=5mm,yshift=-3mm},
boxed title style = {colframe = black!35!black, colback = white},
coltitle = black,
colback = white,
colframe = blue!35!black,
fonttitle = \bfseries,
breakable = true,
top = 4mm
}{korehatheorem3}

\newtcbtheorem{example4}{Example}{enhanced,
attach boxed title to top left = {xshift=5mm,yshift=-3mm},
boxed title style = {colframe = black!35!black, colback = white},
coltitle = black,
colback = white,
colframe = black!35!black,
fonttitle = \bfseries,
breakable = true,
top = 4mm
}{korehatheorem4}

\begin{document}

\setlength{\abovedisplayskip}{1pt}
\setlength{\belowdisplayskip}{1pt}

\onehalfspacing

\title{Topological Space}
\author{}
\date{}
\maketitle

\newpage

\section{Distance}
Let $x,y,z$ be elements of a set $X$ and $d(x,y)$ be a real number that
satisfies conditions below. 
\begin{enumerate}
  \item $d(x,y) \geq 0$ and $d(x,y)=0 \Leftrightarrow x=y$
  \item $d(x,y)=d(y,x)$
  \item $d(x,z) \leq d(x,y)+d(y,z)$
\end{enumerate}
Then, $d$ is called adistance function of $X$ and a set $X$ with a distance
function is called a distance space. $d(x,y)$ is called a distance between $x$
and $y$.

Let $X$ be a distance space, $x \in X$ and $\varepsilon > 0$. Then,
\begin{equation}
U(x,\varepsilon)=U_{\varepsilon}(x)=\{y \in X \ | \ d(x,y)<\varepsilon\}
\end{equation}
is called a $\varepsilon$-neighbourhood and
\begin{equation}
S(x,\varepsilon)=\{y \in X \ | \ d(x,y)=\varepsilon\}
\end{equation}
is called a sphere.a

\begin{example4}{Distance in $R$}{tag}
The distance between $x$ and $y$ which are placed on a line $R$ is denoted by
\begin{equation}
d(x,y)=|x-y|
\end{equation}
\end{example4}

\begin{example4}{Distance in $R^n$}{tag}
The distance between $x=(x_1,x_2,\ldots x_n)$ and $y=(y_1,y_2,\ldots ,y_n)$
which are placed on a line $R^n$ is denoted by
\begin{equation}
d(x,y)=\sqrt{(x_1-y_1)^2+\ldots+(x_n-y_n)^2}
\end{equation}
\end{example4}
This $d(x,y)$ is called Euclid distance and $R^n$ with the distance is called
$n$-dimension Euclid space.  

\begin{example4}{Hilbert Space}{tag}
Hilbert space, which is an extension of an Euclidian space $R^n$ to infinity
dimention is defined as below.
\begin{equation}
R^{\infty}=\{x=(x_1,x_2,\ldots , x_n, \ldots ) \ | \ \sum_{i=1}^{\infty}x_i^2 <
\infty\}
\end{equation}
The distance between two points $x=(x_1,\ldots , x_n, \ldots )$ and
$y=(y_1,\ldots , y_n, \ldots )$ is denoted by 
\begin{equation}
d(x,y)=\sqrt{\sum_{i=1}^{\infty}(x_i-y_i)^2}
\end{equation}
This is finite since for arbitary $n$,
\begin{equation}
\sum_{i=1}^n x_i^2 +\sum_{i=1}^n y_i^2 \geq \sum_{i=1}^n (x_i-y_i)^2
\end{equation}
Both, $\sum_{i=1}^n x_i^2$ and $\sum_{i=1}^n y_i^2$ converge from the
definition, do right-hand side also converges.
\end{example4}

\begin{example4}{Discrete Distance Space}{tag}
For 2 elements $x,y$ of a set $X$ define
\begin{align*}
d(x,x)=0 \\
d(x,y)=1
\end{align*}
This $d(x,y)$ is a distance function, $X$ forms a distance space and this is
called a discrete distance space.
Its $\varepsilon$-neighbourhood is
\begin{align} \displaystyle   
U(x,\varepsilon) =\left\{ 
    \begin{aligned}
    & \{x\} \ (\varepsilon \leq 1)\\
    & X \ (\varepsilon > 1)
    \end{aligned}
\right.
\nonumber
\end{align}
Its open ball is 
\begin{align} \displaystyle   
S(x,\varepsilon) =\left\{ 
    \begin{aligned}
    & \emptyset \ (\varepsilon \neq 1)\\
    & X-\{x\} \ (\varepsilon = 1)
    \end{aligned}
\right.
\nonumber
\end{align}
\end{example4}
For a subset $A$ of a distance space $X$,
\begin{equation}
\delta (A)=\sup \{d(x,y) \ | \ x \in A , y \in A\}
\end{equation}
is called a diameter of $A$.
If diameter of $A$ is finite, then a set $A$ is bounded.

Let $A,B$ be subset of a distance space $X$. Then, the distance between $A$ and
$B$ is denoted by
\begin{equation}
d(A,B)=\inf \{d(x,y) \ | \ x \in A, y \in B\}
\end{equation}

\end{document}